Android Gingerbread on ZedBoard was officially supported by Xilinx partner
company Iveia, unfortunately it is not any more and they do not longer provide
android sources. Thankfully, a github user made the Iveia sources available by
uploading them~\cite{androidsource}.

Android Gingerbread requires a very delicate build environment:
\begin{itemize}
	\item Ubuntu 12.04 LTS host operating system
	\item make version 3.81
	\item gcc, g++ and cpp version 4.4
	\item Java JDK 1.6 from Oracle (the OpenJDK version is not suitable)
	\item The following packages available from the Ubuntu repositories:
		apt-get install libgtk2.0-0:i386 libxtst6:i386 gtk2-engines-murrine:i386
		lib32stdc++6 libxt6:i386 libdbus-glib-1-2:i386 libasound2:i386 fakeroot
		build-essential crash kexec-tools makedumpfile kernel-wedge git-core
		libncurses5 libncurses5-dev libelf-dev asciidoc binutils-dev curl
		gcc-4.4 g++-4.4 cpp-4.4 gcc-4.4-multilib g++-4.4-multilib cpp-4.4
		ia32-libs openjdk-6-jdk:i386 vim gnupg flex bison gperf build-essential
		zip curl libc6-dev x11proto-core-dev libx11-dev:i386
		libreadline6-dev:i386 libgl1-mesa-glx:i386 libgl1-mesa-dev g++-multilib
		mingw32 tofrodos python-markdown libxml2-utils xsltproc zlib1g-dev:i386
		gnupg flex bison gperf build-essential zip curl libc6-dev
		lib32ncurses5-dev x11proto-core-dev libx11-dev:i386
		libreadline6-dev:i386 libgl1-mesa-glx:i386 libgl1-mesa-dev
		g++-multilib mingw32 tofrodos python-markdown libxml2-utils xsltproc
		zlib1g-dev:i386 genext2fs lib32z1-dev
	\item The repo tool (a wrapper around git), available from\\
		\url{http://commondatastorage.googleapis.com/git-repo-downloads/repo}
\end{itemize}
The build process can be seen in \Cref{lst:androidbuild}.
\begin{lstlisting}[
	language=Bash,
	caption={Retrieve and build android},
	label={lst:androidbuild},
	basicstyle=\small,
	float=htbp,
	floatplacement=htbp
	]
repo init -u git://github.com/aimeemikaelac/xilinx-android-manifest.git -b android-zynq-1.0(*@\label{lst:androidret1}@*)
repo sync(*@\label{lst:androidret2}@*)

. build/envsetup.sh(*@\label{lst:androidenv1}@*)
lunch generic-eng(*@\label{lst:androidenv2}@*)


make -j32(*@\label{lst:androidmake1}@*)
make -f Makefile.zynq(*@\label{lst:androidmake2}@*)

mkdir -p root(*@\label{lst:touchcopy1}@*)
sudo mount root.img -o loop,rw,sync root
sudo mkdir -p root/system/usr/idc/
sudo cp <repo>linux-files/touchscreen_config.idc root/system/usr/idc/Vendor_222a_Product_0001.idc
#sudo cp <repo>/linux-files/touchscreen_config.idc "root/system/usr/idc/ILITEK ILITEK-TP.idc"
sudo chmod 664 root/system/usr/idc/*.idc
sleep 1
sudo umount root
sleep 1
rmdir root(*@\label{lst:touchcopy2}@*)
\end{lstlisting}
In \Crefrange{lst:androidret1}{lst:androidret2} the source code is retrieved.
\Crefrange{lst:androidenv1}{lst:androidenv2} setup the environment and
\Crefrange{lst:androidmake1}{lst:androidmake2} builds the system.
The result is a bootable image called `root.img'.

This image then needs to be mounted, so we can copy the touchscreen
configuration file to it. This is done in
\Crefrange{lst:touchcopy1}{lst:touchcopy2}.
