When all the software components described above have been generated, the boot image can finally be created. For this purpose, a \gls{bif} file was created. It tells the bootgen tool in which destination the component should be in the boot image and what type it is. The content of the \gls{bif} file can be seen in \cref{lst:biffile}.
\begin{lstlisting}[
	language=Bash,
	caption={BIF file for the boot image},
	label={lst:biffile},
	basicstyle=\small,
	float=htbp,
	floatplacement=htbp
	]
the_ROM_image:
{
  [destination_cpu=a53-0, bootloader] fsbl.elf
  [pmufw_image] pmufw.elf
  [destination_device=pl] zcu102_wrapper.bit
  [destination_cpu=a53-0, exception_level=el-3, trustzone] bl31.elf
  [destination_cpu=a53-0, exception_level=el-2] u-boot.elf
}
\end{lstlisting}

To finally generate the boot image, run the following command:
\begin{lstlisting}[
	language=Bash,
	%caption={Generating boot image},
	%label={lst:makebootimage},
	basicstyle=\small,
	float=htbp,
	floatplacement=htbp
	]
bootgen -arch zynqmp -image bootimage/zcu102.bif -w -o i bootimage/BOOT.BIN
\end{lstlisting}

The generated boot image will be available at \emph{bootimage/BOOT.BIN}.