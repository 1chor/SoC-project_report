Mentor Embedded also provides a branch of the u-boot utility that is compatible for this setup \cite{uboot}. Source release \emph{release_android-8_xilinx-v2018.1} was also used here so that they are on the same version.

Before the bootloader can be created, a fix for the \gls{pmufw} version check must be implemented. The \gls{pmufw} described in \cref{sssec:pmufw} generates the firmware with version $1.0$, but the u-boot utility checks for version $0.3$. Therefore the version control must be corrected.

To build the u-boot utility and also apply the \gls{pmufw} version fix, the following lines need to be processed:
\begin{lstlisting}[
	language=Bash,
	%caption={Building the u-boot utility},
	%label={lst:makeuboot},
	basicstyle=\small,
	float=htbp,
	floatplacement=htbp
	]
# apply PMUFW version fix
cp build-files/u-boot/cpu.c mpsoc-u-boot-xlnx/arch/arm/cpu/armv8/zynqmp

# load config and build
cd mpsoc-u-boot-xlnx
make CROSS_COMPILE=aarch64-linux-gnu- xilinx_zynqmp_zcu102_rev1_0_defconfig
make CROSS_COMPILE=aarch64-linux-gnu-
\end{lstlisting}

\note{check formatting, need to be under listing}
The compiled bootloader will be available in the u-boot folder as \emph{u-boot.elf}.