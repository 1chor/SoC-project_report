The steps for the devicetree are similar to those for \gls{fsbl} and \gls{pmufw}, but require additional steps.

First, a new devicetree application project is created with the SDK. Xilinx provides a devicetree repository \cite{devicetree} that is required for the application project. The source revision \emph{xilinx-v2018.1} was selected because all components use this tool version. The application project automatically searches the hardware design and adds all necessary files for the device tree.

Since the reconfigurable region is not connected within the block design, but is connected to the exported signals in the wrapper file (see \cref{sssec:partialreconfigurationsetup}), not all of the information required for the reconfigurable modules is contained in the \gls{pl}-related \gls{dtsi} file. Therefore the \emph{pl.dtsi} file has to be adapted.

Finally, the devicetree can be generated with the devicetree generator contained in the Linux kernel sources \cite{linuxkernel}.

To complete these steps, the following commands must be run:
\begin{lstlisting}[
	language=Bash,
	%caption={Compiling the devicetree},
	%label={lst:makedevicetree},
	basicstyle=\small,
	float=htbp,
	floatplacement=htbp
	]
# create devicetree project
cd hardware_design
make -f scripts/Makefile dts

# load pl.dtsi
cp ../build-files/devicetree/pl.dtsi soc_project.sdk/device-tree/

# compiling a devicetree blob
make -f scripts/Makefile compile
\end{lstlisting}

\note{check formatting, need to be under listing}
The compiled devicetree blob will be available at \emph{soc_project.sdk/device-tree/system-top.dtb}, but will be renamed to \emph{zynqmp-zcu102-rev1.0.dtb}.