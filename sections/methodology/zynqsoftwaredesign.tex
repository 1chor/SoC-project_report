The Zynq Software Design consists of the following components:
\begin{itemize}
	\item The Linux kernel
	\item The Linux kernel modules
	\item The \emph{u-boot} utility
	\item The \gls{fsbl}
	\item The \gls{pmufw}
	\item The devicetree
	\item The ARM Trusted Firmware
	\item The boot image
	\item The Android Operating System
	\item The image processing app
\end{itemize}

The following sections are organised as follows: In \cref{sssec:linuxkernel} the steps for the Linux kernel are presented, \cref{sssec:linuxkernelmodules} talks about the kernel modules. The u-boot utility is described in detail in \cref{sssec:uboot}. The steps for the \gls{fsbl} are shown in \cref{sssec:fsbl} while for the \gls{pmufw} they are shown in \cref{sssec:pmuf}. The compilation of the devicetree is presented in \cref{sssec:devicetree}. Setting up the ARM Trusted Firmware is described in \cref{sssec:atf}. The generation of the boot image is explained in \cref{sssec:bootimage}. The Android setup itself is discussed in \Cref{sssec:androidonzcu102}. \Cref{sssec:imageprocessingapp} details the Android app. Finally, \Cref{sssec:dynamicpartialreconfiguration} talks about how we set up
the software system to be able to do \gls{dpr}.
