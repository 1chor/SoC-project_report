Since the previous project \cite{oldrepo} had some complications with the low Android $2.3$ Gingerbread version, a newer version was used. Mentor Embedded provides the source files for Android $8$ Oreo for Xilinx Zynq UltraScale+ MPSoC \cite{androidsource}. 

The Android Oreo setup requires the following build environment:
\begin{itemize}
	\item Ubuntu 18.04 LTS host operating system (tested)
	\item OpenJDK 8
	\item The following packages available from the Ubuntu repositories:
		sudo apt-get install git-core gnupg flex bison build-essential zip curl zlib1g-dev gcc-multilib g++-multilib libc6-dev-i386 lib32ncurses5-dev x11proto-core-dev libx11-dev lib32z1-dev libgl1-mesa-dev libxml2-utils xsltproc unzip fontconfig dosfstools e2fsprogs parted
	\item The repo tool (a wrapper around git), available from\\
		\url{https://storage.googleapis.com/git-repo-downloads/repo}
\end{itemize}

The build process can be seen in \cref{lst:androidbuild}.
\begin{lstlisting}[
	language=Bash,
	caption={Retrieve and build android},
	label={lst:androidbuild},
	basicstyle=\small,
	float=htbp,
	floatplacement=htbp
	]
# fetching android source
repo init -u git://github.com/MentorEmbedded/mpsoc-manifest.git -b zynqmp-android_8 -m release_android-8_xilinx-v2018.1.xml (*@\label{lst:androidret1}@*)
repo sync -c (*@\label{lst:androidret2}@*)

# fetching MALI 400 userspace binaries
mkdir -p ../android/vendor/xilinx/zynqmp/proprietary (*@\label{lst:mali1}@*)
cp -r mali/* ../android/vendor/xilinx/zynqmp/proprietary (*@\label{lst:mali2}@*)

# updating selinux policy files
cp build-files/sepolicy/sv_startup.te android/device/xilinx/zcu102/sepolicy/ (*@\label{lst:sepolicy1}@*)
cp build-files/sepolicy/file_contexts android/device/xilinx/zcu102/sepolicy/ (*@\label{lst:sepolicy2}@*)

# prepare client app
mkdir packages/apps/SoC (*@\label{lst:app1}@*)
cp ../client/app/build/outputs/apk/debug/app-debug.apk packages/apps/SoC/SoC_Client.apk
cp ../build-files/app/Android.mk packages/apps/SoC/
cp ../build-files/app/core.mk build/target/product/ (*@\label{lst:app2}@*)

# making android
source build/envsetup.sh (*@\label{lst:androidenv1}@*)
lunch zcu102-eng (*@\label{lst:androidenv2}@*)
make -j (*@\label{lst:androidmake}@*)
cp out/target/product/zcu102/ramdisk.img ../build-files/ramdisk (*@\label{lst:ramdisk1}@*)

# modify ramdisk.img
./modify_ramdisk.sh extract (*@\label{lst:ramdiskextract}@*)
cp init.rc ramdisk/ (*@\label{lst:ramdiskstartup}@*)
mkdir -p ramdisk/lib/firmware (*@\label{lst:ramdiskpr}@*)
.modify_ramdisk.sh wrap (*@\label{lst:ramdisk2}@*)
cp uramdisk.img ../../bootimage
\end{lstlisting}

In \crefrange{lst:androidret1}{lst:androidret2} the android source code is fetched.
\Crefrange{lst:mali1}{lst:mali2} load the MALI 400 userspace binaries provided by Xilinx \cite{mali400}. 
In order to automatically load the kernel modules described in \cref{sssec:linuxkernelmodules} and configure the \gls{adb}, the ramdisk has to be adapted to run a startup script. In order for the startup script to be executed at all, the selinux policy must be extended to include the startup service. This is done in \crefrange{lst:sepolicy1}{lst:sepolicy2}.
\Crefrange{lst:app1}{lst:app2} prepare the client app described in \cref{sssec:imageprocessingapp} for pre-installation on the android.
\Crefrange{lst:androidenv1}{lst:androidenv2} setup the environment and
\cref{lst:androidmake} builds the android system.
The result are two images called \emph{ramdisk.img} and \emph{system.img}.

The modification of the ramdisk is done in the \crefrange{lst:ramdisk1}{lst:ramdisk2}. In \cref{lst:ramdiskextract} the ramdisk is extracted to a folder. The new \emph{init.rc} file, which contains the startup service, is loaded into the ramdisk folder in \cref{lst:ramdiskstartup}. The preparation for the partial reconfiguration is done in \cref{lst:ramdiskpr}. This folder is needed for the \gls{fpga} manager (see \cref{sssec:dynamicpartialreconfiguration}). \Cref{lst:ramdisk2} wraps the modified ramdisk. The result is a bootable image called \emph{uramdisk.img}.
