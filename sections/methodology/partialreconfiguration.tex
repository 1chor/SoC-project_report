The branch of the Linux kernel provided by Mentor Embedded \cite{linuxkernel} already included a device driver for partial reconfiguration.

A folder must be created before the driver can be used. This adaptation is described in \cref{sssec:androidonzcu102}. 

From Android, \gls{dpr} can be easily done by sending the partial bitstream to the device driver. \Cref{lst:prandroid} shows how this is implemented.
In \cref{lst:flag}, the flag for partial reconfiguration is set so that the \gls{fpga} manager knows the incoming bitstream is a partial one. \Cref{lst:copybit} copies the bitstream from the application folder to a temporary folder. The \gls{fpga} manager can only process bitstreams contained in this directory. Finally, in \cref{lst:prbit}, the partial reconfiguration is performed by passing the bitstream to the \gls{fpga} manager.

\begin{lstlisting}[
    language=Bash,
    caption={Partial reconfiguration controlled by Android},
    label={lst:prandroid},
    basicstyle=\small,
    float=htbp,
    floatplacement=htbp
    ]
# set flags for partial reconfiguration
echo 1 > /sys/class/fpga_manager/fpga0/flags  (*@\label{lst:flag}@*)

# load partial bitstream
cp <path>/bitstream.bin /lib/firmware  (*@\label{lst:copybit}@*)
echo bitstream.bin > /sys/class/fpga_manager/fpga0/firmware  (*@\label{lst:prbit}@*)
\end{lstlisting}

After the reconfiguration is completed, the new logic part can be used.