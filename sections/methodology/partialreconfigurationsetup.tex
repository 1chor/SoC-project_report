In contrast to the previous project \cite{oldrepo}, partial reconfiguration was done using the tools included in the Vivado Design suite. Compared to the planAhead tool included in the Xilinx ISE Design Suite, Vivado can process the \gls{dpr} with the \gls{vhdl} files instead of the synthesised netlists.

For \gls{dpr} a Bottom-Up Synthesis is required. The static logic is synthesised with a black box module definition for each \gls{dpr} module. All reconfigurable modules and the used ip cores in the block design are synthesised using the Out-of-Context module runs.

The bitstream generation is done inside the script\\\emph{<repo>/bootimage/generate\_without\_android.sh}. The synthesis of the Out-of-Context modules and the static logic, the implementation of the different configurations (different filter logics) and the bitstream generation are done with the following line:

\begin{lstlisting}[
	language=Bash,
	%caption={Bitstream generation for project},
	%label={lst:makebit},
	basicstyle=\small,
	float=htbp,
	floatplacement=htbp
	]
# generate the hardware design and the bitstreams
make -f scripts/Makefile bit
\end{lstlisting}

This command runs an underlying \gls{tcl} script that checks the status of the current hardware design. Only when synthesis, implementation or bit stream generation is required, the appropriate steps are taken.

As described in \cite{UG909} the IP integrator does not support \gls{dpr}. This means that \gls{rp} definitions are not yet supported in block diagrams. To overcome this missing feature, all \glspl{rp} have to be defined within \gls{rtl} or the block diagrams have to be exported to the scripted non-project implementation flow. 

In order to add a \gls{rp} to the block design, the required signals were exported. In this case the \gls{axi} interface must be exported:
\begin{itemize}
    \item MO5\_AXI (AXI-Lite interface)
    \item s\_axi\_aclk (Clock)
    \item peripheral\_aresetn[0:0] (Reset)
\end{itemize}

The \gls{rp} IP core must be integrated into the wrapper file for the block diagram generated by Vivado. The exported signals from the block design can now be connected to the IP core there.

Xilinx provides a tutorial \cite{UG947} that describes how the \gls{dpr} is done with Vivado. 

First, the project must be prepared for the partial reconfiguration design flow by enabling Partial Reconfiguration. Once this is set it cannot be undone. Now the partition definitions can be created and the reconfigurable modules can be defined. This step also creates a Out-of-Context module run for each reconfigurable module and creates the different configuration runs.

Each \gls{rp} is required to have a physical block, called Pblock, to define the physical resources available for the reconfigurable module. Pblocks are explicit physical constraints used to define allowable placement areas for specified logic. The Pblock for this design is defined inside the file\\\emph{<repo>/hardware\_design/src/constraints/pblocks.xdc}

In our project $6$ bitstreams are generated:
\begin{itemize}
	\item $3$ full bitstreams with different filter logic
	\item $3$ partial bitstreams with different filter logic
\end{itemize}

Before the partial bitstreams can be used, they need to be transformed into an other format. A \gls{bif} file was created for each reconfigurable module. This file defines the configuration for the Vivado bootgen tool. With the following lines the bitstreams are converted:

\begin{lstlisting}[
    language=Bash,
    %caption={Convert bitstreams into binary files},
    %label={lst:convertbit},
    basicstyle=\small,
    float=htbp,
    floatplacement=htbp
]
cd hardware_design/generated_bitstreams
bootgen -image blue_filter.bif -arch zynqmp -o ../../server/downloads/blue_filter.bin -w
bootgen -image green_filter.bif -arch zynqmp -o ../../server/downloads/green_filter.bin -w
bootgen -image red_filter.bif -arch zynqmp -o ../../server/downloads/red_filter.bin -w
\end{lstlisting}

The generated binary bitstreams are copied to the download folder of the server (see section \ref{ssssec:server}).